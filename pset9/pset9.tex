\documentclass[11pt,oneside]{amsart}
\usepackage{geometry}
\usepackage{amssymb,parskip,mathtools,microtype}
\usepackage[shortlabels]{enumitem}

\theoremstyle{definition}
\newtheorem{problem}{Problem}

\newcommand{\bC}{\mathbb{C}}
\newcommand{\bF}{\mathbb{F}}
\newcommand{\bN}{\mathbb{N}}
\newcommand{\bQ}{\mathbb{Q}}
\newcommand{\bR}{\mathbb{R}}
\newcommand{\bZ}{\mathbb{Z}}
\newcommand{\eps}{\varepsilon}
\newcommand*\colvec[1]{\begin{psmallmatrix}#1\end{psmallmatrix}}
\newcommand*\dcolvec[1]{\begin{pmatrix}#1\end{pmatrix}}

\DeclareMathOperator{\Span}{span}
\let\Re\relax
\DeclareMathOperator{\Re}{Re}
\let\Im\relax
\DeclareMathOperator{\Im}{Im}
\DeclareMathOperator{\im}{im}
\DeclareMathOperator{\rank}{rank}

\title{MATH2211 Spring 2022\\
Problem Set 9}
\author{Due Friday, April 15, 2022 at 11:59 pm}

\begin{document}
    \maketitle

    Useful reading for Problems 1 and 2: Section 8.A of Axler.

    \begin{problem}
        A \emph{nilpotent} linear operator is defined to be a linear operator $T\colon V\to V$ such that some power of $T$ is equal to zero. In parts (b) to (d), let $N\colon V\to V$ be a nilpotent operator on a finite dimensional vector space $V$.
        \begin{enumerate}[(a)]
            \item Give an example of a $2\times 2$ real nilpotent matrix (i.e.\ a nilpotent linear operator from $\bR^2$ to $\bR^2$) none of whose entries are 0 (or prove they don't exist).
            \item Prove that 0 is the only eigenvalue of $N$.
            \item Let $U$ be any nonzero subspace of $V$ and suppose that $NU\subseteq U$. Prove that $NU$ is strictly contained in $U$.
            
            Hint: Use contradiction. Comment: A useful notation for strict containment is $\subsetneq$.
            \item Prove that $N^{\dim V}=0$. Hint: Use the previous part iteratively starting with $U=V$.
        \end{enumerate}
    \end{problem}
    
    \begin{problem}
        Given a polynomial $p\in F[t]$ (note: $F[t]$ just means the set of polynomials in the variable $t$ with coefficients in $F$) and a linear operator $T\colon V\to V$ on a vector space $V$ over $F$, the expression $p(T)$ makes sense if we use powers and addition of linear operators, to give a linear operator $p(T)\colon V\to V$.

        \begin{enumerate}[(a)]
            \item Let $V=F^2$ and $T\colvec{x\\y}=\colvec{x+y\\y}$. For the polynomial $p(t)=t^2-2t$, find $p(T)$.
            \item Prove that for every linear operator $T\colon V\to V$ with $\dim V=n$, there exists a polynomial $p\in F[t]$, of degree at most $n^2$, such that $p(T)=0$.
            
            Note: Do not use the theory of minimal polynomials or any theorems named after multiple people.

            Hint: Use linear dependence ideas.
            \item Let $J_{n,\lambda}$ be the $n\times n$ Jordan block (this is the standard name for what I called an ``atomic Jordan matrix'' in class) with $\lambda$'s on the diagonal and 1's above the diagonal. Prove that $p(J_{n,\lambda})=0$ for the polynomial $p(t)=(t-\lambda)^n$, but $q(J_{n,\lambda})\neq 0$ for the polynomial $q(t)=(t-\lambda)^{n-1}$.
            
            Comment (not a hint): This problem essentially says that $J_{n,\lambda}-\lambda I_n$ is a nilpotent operator and that $n$ is the least power $k$ that makes $(J_{n,\lambda}-\lambda I_n)^k=0$.
        \end{enumerate}
    \end{problem}
    
    \begin{problem}
        A \emph{permutation matrix} is a square matrix where in each column and each row, there is exactly one nonzero entry and that nonzero entry is a 1. The name is because an $n\times n$ permutation matrix times an $n\times 1$ vector is the same vector but with entries permuted. Let $P$ be an $n\times n$ permutation matrix.
        \begin{enumerate}[(a)]
            \item Prove that the product of two $n\times n$ permutation matrices is another permutation matrix.
            \item Using part (a), prove that some positive power of $P$ is equal to the identity.
            
            Hint: Are there infinitely many $n\times n$ permutation matrices?
            \item Provide a counterexample to the claim that for all $n\in\bZ^+$ and all $n\times n$ permutaion matrices $P$, there is some $1\leq k\leq n$ such that $P^k$ is the identity.
            
            Hint: You will not find a counterexample by looking at $n\leq 4$.
        \end{enumerate}
    \end{problem}

    \begin{problem}
        Let $P_3$ be the space of real polynomials of degree at most 3, and define an inner product on $P_3$ by
        \[\langle f,g\rangle=\int_0^1 f(x)g(x)\,dx.\]
        Set $U=\Span\{x,x^2\}$.
        \begin{enumerate}[(a)]
            \item Find an orthonormal basis for $U$.
            \item Write $x^3=p(x)+q(x)$ with $p(x)\in U$ and $q(x)\in U^\perp$.
        \end{enumerate}
    \end{problem}
\end{document}
