\documentclass[12pt]{article}
\usepackage[T1]{fontenc}
\usepackage{lmodern}
\usepackage{geometry}
\usepackage{amsmath,amssymb,amsthm}

\newtheorem{theorem}{Theorem}

\newcommand{\bR}{\mathbb{R}}

\title{Closed form for Fibonacci Sequence (advertisement for vector spaces and eigendecomposition)}
\author{MATH2212 Spring 2022}
\date{January 26, 2022}
\begin{document}
\maketitle

\begin{theorem}[Binet's formula]
    Let $F_0=0,F_1=1$, and $F_n=F_{n-1}+F_{n-2}$ for $n\geq 2$. Now let $\phi=(1+\sqrt5)/2$. Then
    \[F_n=\frac 1{\sqrt5}\phi^n-\frac 1{\sqrt5}\left(-\frac 1{\phi}\right)^n\quad\text{for all }n\geq 0.\]
\end{theorem}
\begin{proof}[Proof using vector spaces]
    Let us consider the set $V$ of all real-valued sequences $(a_n)_{n=0}^\infty$ satisfying $a_n=a_{n-1}+a_{n-2}$ for $n\geq 2$. We can see that $V$ is a vector space over $\bR$ because the sum of two sequences satisfying the recurrence still satisfies the recurrence, and any scalar multiple of a sequence satisfying the recurrence still satisfies the recurrence. Moreover, a sequence in $V$ is determined by its first two terms $a_0$ and $a_1$, which can be freely chosen, so $V$ is a 2-dimensional vector space over $\bR$.
    
    Let $\phi$ be the positive root of $x^2-x-1$ (i.e.\ the golden ratio), so its negative root is $-1/\phi$. The sequences $(1,\phi,\phi^2,\dots)$ and $(1,-1/\phi,(-1/\phi)^2,\dots)$ are both in $V$ and are linearly independent, and therefore form a basis of $V$ since $V$ is 2-dimensional. Hence $(F_0,F_1,F_2,\dots)=c_1(1,\phi,\phi^2,\dots)+c_2(1,-1/\phi,(-1/\phi)^2,\dots)$ for some $c_1$ and $c_2$. This yields an infinite sequence of linear equations involving $c_1$ and $c_2$. Using the first two of these to solve for $c_1$ and $c_2$ we get $c_1=\frac 1{\sqrt5}$ and $c_2=-\frac 1{\sqrt5}$, thus obtaining the formula
    \[F_n=\frac 1{\sqrt5}\phi^n-\frac 1{\sqrt5}\left(-\frac 1{\phi}\right)^n\quad\text{for all }n\geq 0.\]
\end{proof}
\newpage
\begin{proof}[A bit more detail on how to solve for $c_1$ and $c_2$]
    In what follows, I have renamed $\phi$ to $r$ and $-1/\phi$ to $s$. As above, the sequences $(1,r,r^2,r^3,\dots)$ and $(1,s,s^2,s^3,\dots)$ are both in $V$ and are linearly independent, so they form a basis of $V$. That is, every element of $V$ is a real linear combination of these two sequences. In particular, $(0,1,1,2,3,\dots)$, the Fibonacci sequence, is a linear combination of $(1,r,r^2,r^3,\dots)$ and $(1,s,s^2,s^3,\dots)$. So there are unique real numbers $c_1$ and $c_2$ such that
    \[(0,1,1,2,3,\dots) = c_1(1,r,r^2,\dots)+c_2(1,s,s^2,\dots).\]
    If we expand what this says, this says that
    \begin{align*}
        F_0&=0 =c_1+c_2\\
        F_1&=1 = c_1r+c_2s\\
        &\vdots \\
        F_n &=c_1r^n+c_2s^n\quad\text{ for all }n\geq 2.
    \end{align*}
    The first two lines allow us to solve for $c_1$ and $c_2$: the first line says $c_2=-c_1$, and this combined with the second line implies that $c_1=\frac 1{r-s}=-\frac 1{\sqrt5}$ (because $r-s=\sqrt5$).
\end{proof}
\newpage
\begin{proof}[Proof by eigendecomposition]
    Let $M=\begin{pmatrix}1&1\\1&0\end{pmatrix}$. Then one can verify (by induction) that
    \[M^n=\begin{pmatrix}F_{n+1}&F_n\\F_n&F_{n-1}\end{pmatrix}\quad\text{for all }n\geq 1.\]
    Now $M$ has eigenvalues $\phi=(1+\sqrt5)/2$ and $-1/\phi=(1-\sqrt5)/2$ because the characteristic polynomial of $M$ is $x^2-x-1$. The matrix $M$ has eigendecomposition
    \[\begin{split}
        M&=\begin{pmatrix}\phi&-\frac 1{\phi}\\1&1\end{pmatrix}\begin{pmatrix}\phi&0\\0&-\frac 1{\phi}\end{pmatrix}\begin{pmatrix}\phi&-\frac 1{\phi}\\1&1\end{pmatrix}^{-1}\\
        &= \frac 1{\sqrt5}\begin{pmatrix}\phi&-\frac 1{\phi}\\1&1\end{pmatrix}\begin{pmatrix}\phi&0\\0&-\frac 1{\phi}\end{pmatrix}\begin{pmatrix}1&\frac 1{\phi}\\-1&\phi\end{pmatrix}.
        \end{split}\]
    The purpose of this eigendecomposition is that powers of $M$ now have a closed form:
    \[\begin{split}
        M^n&=\begin{pmatrix}\phi&-\frac 1{\phi}\\1&1\end{pmatrix}\begin{pmatrix}\phi^n&0\\0&\left(-\frac 1{\phi}\right)^n\end{pmatrix}\begin{pmatrix}\phi&-\frac 1{\phi}\\1&1\end{pmatrix}^{-1}\\
        &= \frac 1{\sqrt5}\begin{pmatrix}\phi&-\frac 1{\phi}\\1&1\end{pmatrix}\begin{pmatrix}\phi^n&0\\0&\left(-\frac 1{\phi}\right)^n\end{pmatrix}\begin{pmatrix}1&\frac 1{\phi}\\-1&\phi\end{pmatrix}\\
        &= \frac 1{\sqrt5}\begin{pmatrix}\phi & -\frac 1{\phi}\\1&1\end{pmatrix}\begin{pmatrix} \phi^n & \phi^{n-1}\\-\left(-\frac1{\phi}\right)^n & \phi\left(-\frac 1{\phi}\right)^n\end{pmatrix}.
        \end{split}\]
    We only want $F_n$, so let's compute the top right entry of the product. We find that it is
    \[\frac 1{\sqrt5}\left(\phi^n-\left(-\frac 1{\phi}\right)^n\right),\]
    as promised.
\end{proof}

\end{document}