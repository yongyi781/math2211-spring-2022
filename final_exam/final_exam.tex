\documentclass[11pt,oneside]{amsart}
\usepackage[margin=1in]{geometry}
\usepackage{amssymb,parskip,mathtools,microtype}
\usepackage[shortlabels]{enumitem}

\theoremstyle{definition}
\newtheorem{problem}{Problem}

\newcommand{\bC}{\mathbb{C}}
\newcommand{\bF}{\mathbb{F}}
\newcommand{\bQ}{\mathbb{Q}}
\newcommand{\bR}{\mathbb{R}}
\newcommand{\bZ}{\mathbb{Z}}
\newcommand*\colvec[1]{\begin{psmallmatrix}#1\end{psmallmatrix}}
\newcommand*\dcolvec[1]{\begin{pmatrix}#1\end{pmatrix}}
\newcommand{\extp}{\mathchoice{{\textstyle\bigwedge}}%
    {{\bigwedge}}%
    {{\textstyle\wedge}}%
    {{\scriptstyle\wedge}}}

\DeclareMathOperator{\Span}{span}
\let\Re\relax
\DeclareMathOperator{\Re}{Re}
\let\Im\relax
\DeclareMathOperator{\Im}{Im}

\title{MATH2211 Spring 2022\\
Final Exam}
\author{Wednesday, April 20 2022}

\begin{document}
    \maketitle

    Name: \underline{\hspace{6cm}}

    This exam is open notes. There are 100 points total in this exam.

    \begin{problem}
        (10 points) Let $C(\bR)$ denote the space of continuous functions from $\bR$ to $\bR$. The Dirac delta distribution $\delta$ is an example of an element of $C(\bR)^*$ and is defined to be the linear functional on $C(\bR)$ sending a function $f\in C(\bR)$ to $f(0)$.

        Let $f(x)=x^2+x+4$. Let $M_f\colon C(\bR)\to C(\bR)$ be the multiplication by $f$ operator, so $M_f(g)=(x^2+x+4)g$ for all $g\in C(\bR)$. Therefore, the transpose of $M_f$ is an operator $^tM_f\colon C(\bR)^*\to C(\bR)^*$. What is
        \[^t M_f(\delta)?\]
    \end{problem}

    \begin{problem}
        (10 points) Give an example of a non-diagonalizable $2\times 2$ 
    \end{problem}
    \vfill
\end{document}
