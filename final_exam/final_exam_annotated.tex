\documentclass[11pt,oneside]{amsart}
\usepackage[margin=1in]{geometry}
\usepackage{amssymb,parskip,mathtools,microtype,xcolor}
\usepackage[shortlabels]{enumitem}

\theoremstyle{definition}
\newtheorem{problem}{Problem}

\newcommand{\bC}{\mathbb{C}}
\newcommand{\bF}{\mathbb{F}}
\newcommand{\bQ}{\mathbb{Q}}
\newcommand{\bR}{\mathbb{R}}
\newcommand{\bZ}{\mathbb{Z}}
\newcommand{\eps}{\varepsilon}
\newcommand*\colvec[1]{\begin{psmallmatrix}#1\end{psmallmatrix}}
\newcommand*\dcolvec[1]{\begin{pmatrix}#1\end{pmatrix}}
\newcommand{\extp}{\mathchoice{{\textstyle\bigwedge}}%
    {{\bigwedge}}%
    {{\textstyle\wedge}}%
    {{\scriptstyle\wedge}}}

\DeclareMathOperator{\Span}{span}
\let\Re\relax
\DeclareMathOperator{\Re}{Re}
\let\Im\relax
\DeclareMathOperator{\Im}{Im}
\let\tr\relax
\DeclareMathOperator{\tr}{tr}

\title{MATH2211 Spring 2022\\
Final Exam}
\author{Friday, May 13 2022}

\definecolor{maroon}{RGB}{204,0,0}

\begin{document}
    \maketitle

    Name: \underline{\hspace{6cm}}

    This exam is open notes, and the time limit is 3 hours. There are 100 points total in this exam.

    {\color{maroon}
    High 87, low 39, mean 70.15, median 71. Good exam. The intended structure was that the first 5 problems were standard linear algebra problems, while the last 5 required at least one step of original thought. The scores reflected this quite well; average score across problems 1 to 5 was 85.5\%, while average score across problems 6 to 10 was 54.8\%. The hardest problems were 7(a), 7(b), 8(b), 9, and 10(b), with average points obtained being 43\%, 44\%, 39\%, 43\%, and 50\%. It also turned out that 8(a) and 10(a) from the second half of the exam were quite easy; average score on those was 80\%.

    The top score on problem 9 was 6/10, which means that the problem should have a substantial hint in the future.

    Nobody got full points on problem 8(b), despite the extensive hint. Two people came close, saying that $I$ and $C^{-1}$ are both in $V_C$. The issue is of course that $C^{-1}$ does not always exist, and they should have used $C$ instead of $C^{-1}$.
    }

    \begin{problem}
        Let $M=\begin{pmatrix}
            1&2&-1\\
            0&4&1\\
            2&-1&0
        \end{pmatrix}$.
        \begin{enumerate}[(a)]
            \item (5 points) Solve the system
            \[M\begin{pmatrix}
                x\\y\\z
            \end{pmatrix}=\begin{pmatrix}
                a\\1\\0
            \end{pmatrix}\]
            in terms of $a$.
            \vfill
            \item (5 points) Find $\tr M$, $\det M$, and the characteristic polynomial of $M$.
            \vfill
        \end{enumerate}
    \end{problem}
    
    \begin{problem}
        (10 points) For which $t\in\bR$ do the vectors
        \[v_1=\dcolvec{1\\0\\t},\quad v_2=\dcolvec{t\\1\\1}, \quad v_3=\dcolvec{0\\1\\2}\]
        form a basis of $\bR^3$?
    \end{problem}
    \vfill
    \begin{problem}
        (10 points) Show that every complex solution to $z^4+1=0$ also satisfies $z^{1200}=1$.
    \end{problem}
    \vfill
    
    \begin{problem}
        % (10 points) Let $V$ be a finite dimensional $\bF_2$-vector space. Prove that the number of elements in $V$ is a power of 2 (i.e.\ $2^k$ for some non-negative integer $k$).

        % Hint: Use a basis of $V$.
        (10 points) Find the inverse of the linear operator $T\colon \bC^3\to\bC^3$ given by
        \[T(x,y,z)=(x+y,x+z,y+z),\]
        or prove that no such inverse exists.
    \end{problem}
    \vfill
    
    \begin{problem}
        (10 points) Find a basis of eigenvectors for the matrix
        \[\begin{pmatrix}0&0&1\\1&0&0\\0&1&0\end{pmatrix}.\]
        Hint: It will be helpful to let $\zeta=e^{2\pi i/3}$. (Make use of the zeta drawing skills you learned in class!)
    \end{problem}
    \vfill
    
    \begin{problem}
        (10 points) Let $X$ be a $2\times 2$ real matrix with trace 0 and rank 1. Prove that $X$ only has 0 as an eigenvalue.
    \end{problem}
    \vfill
    
    \begin{problem}
        Let $V=C([0,1],\bR)$ be the space of continuous functions from $[0,1]$ to $\bR$. The integral operator $I$ defined by
        \[(I(f))(x)=\int_0^x f(t)\,dt\]
        is a linear operator on $V$.
        \begin{enumerate}[(a)]
            \item (5 points) Prove that $I$ is not surjective.
            
            Hint: The non-surjectivity comes from a simple observation; no real analysis knowledge is required.
            \vfill
            \item (5 points) What is $(^tI)(\delta)$, where $\delta$ is the Dirac delta functional? Show that your answer gives another proof that $I$ is not surjective.
            \vfill
        \end{enumerate}
    \end{problem}
    
    \begin{problem}
        Let $n\geq 2$ be a positive integer. For any matrix $C\in M_n(\bR)$, let $V_C$ be the set of all $n\times n$ matrices $A\in M_n(\bR)$ such that $AC=CA$.
        \begin{enumerate}[(a)]
            \item (5 points) Prove that $V_C$ is a subspace of $M_n(\bR)$.
            \vfill
            \item (5 points) Prove that $\dim V_C\geq 2$ for all $C\in M_n(\bR)$.

            Hint: Consider the case when $C$ is a scalar matrix and the case when $C$ is not a scalar matrix separately. In the latter case, try to come up with two linearly independent matrices in $V_C$.
            \vfill
        \end{enumerate}
    \end{problem}
    
    \begin{problem}
        (10 points) Find a counterexample to the following (reasonable-sounding) claim: If $P$ and $Q$ are orthogonal projection operators, then $PQ$ is also an orthogonal projection operator.
    \end{problem}
    
    % \begin{problem}
    %     (10 points) Let $C(\bR)$ denote the space of continuous functions from $\bR$ to $\bR$. Let $\varphi\colon C(\bR)\to C(\bR)$ be the linear map sending an element $f(x)\in C(\bR)$ to $(x^2+x+4)\cdot f(x)$. Prove that
    %     \[(^t \varphi)(\delta)=4\delta,\]
    %     where $\delta$ is the Dirac delta functional sending $f\in C(\bR)$ to $f(0)$.

    %     Hint: Follow your nose carefully through the definitions.
    % \end{problem}
    
    \begin{problem}
        \leavevmode\begin{enumerate}[(a)]
            \item (5 points) Let $e_1\dots e_n$ be an orthonormal basis of an inner product space $V$. Prove that the functionals $\eps_i\in V^*$ defined by
            \[\eps_i(x)=\langle x, e_i\rangle, \quad 1\leq i\leq n,\]
            are precisely the dual basis of $e_1,\dots,e_n$.
            \vfill

            \item (5 points) Suppose now that $e_1,\dots,e_n$ is only an orthogonal basis, meaning that $\langle e_i,e_j\rangle=0$ for $i\neq j$, but the $\|e_i\|$ are not necessarily equal to 1. For each $i$, let $\ell_i=\|e_i\|$, the length of $e_i$. Find the dual basis to $e_1,\dots,e_n$, in terms of the $\eps_i$ defined in part (a) as well as the $\ell_i$.
            \vfill
        \end{enumerate}
    \end{problem}
\end{document}
