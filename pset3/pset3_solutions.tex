\documentclass[11pt,oneside]{amsart}
\usepackage{geometry}
\usepackage{amssymb,parskip,mathtools,microtype}
\usepackage[shortlabels]{enumitem}
\usepackage[most]{tcolorbox}

\theoremstyle{definition}
\newtheorem{problem}{Problem}

\definecolor{sol}{rgb}{0.1, 0.3, 0.6}

\newtcolorbox{solution}{enhanced, breakable, colframe=sol, title=Solution}
\newtcolorbox{rubric}{colframe=purple, title=Rubric}

\newcommand{\bC}{\mathbb{C}}
\newcommand{\bF}{\mathbb{F}}
\newcommand{\bQ}{\mathbb{Q}}
\newcommand{\bR}{\mathbb{R}}
\newcommand{\bZ}{\mathbb{Z}}
\DeclareMathOperator{\Span}{span}

\title{MATH2211 Spring 2022\\
Problem Set 3 Solutions}

\begin{document}
    \maketitle

    Let $F$ be a field and let $V$ be an $F$-vector space.

    \begin{problem}
        Suppose we are given a list $v_1,\dots,v_n\in V$.
        \begin{enumerate}[(a)]
            \item Show that $v_1,\dots,v_n$ are linearly dependent if and only if there is some $1\leq i\leq n$ such that $v_i\in\Span(v_1,\dots,v_{i-1},v_{i+1},\dots,v_n)$ (i.e.\ the span of the list with $v_i$ taken out).
            \begin{solution}
                First suppose that $v_1,\dots,v_n$ are linearly dependent. One characterization of linear dependence is that some vector $v_i$ is a linear combination of the other vectors $v_1,\dots,v_{i-1},v_{i+1},\dots,v_n$. Therefore, for this choice of $i$, $v_i\in\Span(v_1,\dots,v_{i-1},v_{i+1},\dots,v_n)$.

                Now suppose that some $v_i$ is in the span of $v_1,\dots,v_{i-1},v_{i+1},\dots,v_n$. This says that the vector $v_i$ is a linear combination of the other vectors in the list. That implies that $v_1,\dots,v_n$ is a linearly dependent set.
            \end{solution}
            \item Show that $v_i\in\Span(v_1,\dots,v_{i-1},v_{i+1},\dots,v_n)$ if and only if
            \[\Span(v_1,\dots,v_n)=\Span(v_1,\dots,v_{i-1},v_{i+1},\dots,v_n).\]
            (That is, the span doesn't change when $v_i$ is taken out.)
            \begin{solution}
                First note that for two subsets $A\subseteq B$ of $V$, $\Span A\leq\Span B$ automatically. So to say that $\Span(v_1,\dots,v_n)=\Span(v_1,\dots,v_{i-1},v_{i+1},\dots,v_n)$ is equivalent to saying that $\Span(v_1,\dots,v_n)\subseteq\Span(v_1,\dots,v_{i-1},v_{i+1},\dots,v_n)$, because the other containment always holds. This in turn is equivalent to saying that every linear combination of $v_1,\dots,v_n$ is also a linear combination of $v_1,\dots,v_{i-1},v_{i+1},\dots,v_n$.

                For one direction, suppose that $v_i\in\Span(v_1,\dots,v_{i-1},v_{i+1},\dots,v_n)$ for some $i$. This says that there exist scalars $a_j\in F$ for $j\in\{1,2,\dots,i-1,i+1,\dots,n\}$ such that $v_i=\sum_{j\in\{1,2,\dots,i-1,i+1,\dots,n\}}a_jv_j$. Now let
                \[v=b_1v_1+\cdots+b_nv_n\]
                be an arbitrary element in $\Span(v_1,\dots,v_n)$. By replacing the $b_iv_i$ term with the sum $\sum_{j\in\{1,2,\dots,i-1,i+1,\dots,n\}}b_ia_jv_j$, we can rewrite $v$ as
                \[v=(b_ia_1+b_1)v_1+\cdots+(b_ia_{i-1}+b_{i-1})v_{i-1}+(b_ia_{i+1}+b_{i+1})v_{i+1}+(b_ia_n+b_n)v_n,\]
                showing that $v$ is in the span of $v_1,\dots,v_{i-1},v_{i+1},\dots,v_n$.

                For the other direction, suppose that every linear combination of $v_1,\dots,v_n$ is also a linear combination of $v_1,\dots,v_{i-1},v_{i+1},\dots,v_n$. Now it is certainly true that $v_i$ is a linear combination of $v_1,\dots,v_n$, so this says that $v_i$ is also able to be written as a linear combination of $v_1,\dots,v_{i-1},v_{i+1},\dots,v_n$. This shows that $v_i\in\Span(v_1,\dots,v_{i-1},v_{i+1},\dots,v_n)$ and concludes the proof.
            \end{solution}
        \end{enumerate}
    \end{problem}
    
    \begin{problem}
        Suppose $v_1,v_2,v_3,v_4\in V$ and set
        \[w_1=v_1-v_2,\quad w_2=v_2-v_3,\quad w_3=v_3-v_4,\quad w_4=v_4.\]
        \begin{enumerate}[(a)]
            \item Show that $\Span(v_1,v_2,v_3,v_4)=\Span(w_1,w_2,w_3,w_4)$.
            \begin{solution}
                As was given, each $w_i$ is a linear combination of the $v_i$. In other words, $w_i\in\Span(v_1,v_2,v_3,v_4)$ for all $i$. Therefore,
                \[\Span(w_1,w_2,w_3,w_4)\leq\Span(v_1,v_2,v_3,v_4)\]
                since $\Span(w_1,w_2,w_3,w_4)$ is the smallest subspace of $V$ containing the vectors $w_1,w_2,w_3,w_4$.

                Now we can also write each $v_i$ as a linear combination of the $w_i$ as follows:
                \[v_1=w_1+w_2+w_3+w_4,\quad v_2=w_2+w_3+w_4,\quad v_3=w_3+w_4,\quad v_4=w_4.\]
                By the same argument as above, this shows that
                \[\Span(w_1,w_2,w_3,w_4)\geq\Span(v_1,v_2,v_3,v_4).\]
                Thus we are done.
            \end{solution}
            \item Show that $v_1,v_2,v_3,v_4$ are linearly independent if and only if $w_1,w_2,w_3,w_4$ are linearly independent.
            \begin{solution}
                Here is a very neat proof for this problem. We first have the following lemma: A finite set of vectors $S\subseteq V$ is linearly independent if and only if the dimension of $\Span (S)$ is equal to $|S|$. Proof: If $S$ is linearly independent, then $S$ is a basis of $\Span (S)$ since $S$ is linearly independent and $S$ spans $\Span (S)$. Conversely, if $\dim\Span S=|S|$, Corollary 4.26 from the course notes along with the fact that $S$ spans $\Span(S)$ shows that $S$ is a basis of $\Span (S)$, and therefore linearly independent. 

                Using this lemma, we have that $v_1,v_2,v_3,v_4$ are linearly independent iff $\Span(v_1,v_2,v_3,v_4)$ has dimension 4, and the same for $w_1,w_2,w_3,w_4$. Since part (a) proved that
                \[\Span(v_1,v_2,v_3,v_4)=\Span(w_1,w_2,w_3,w_4),\]
                it follows that one side has dimension 4 iff the other side has dimension 4. By the lemma, this is the same as saying that $v_1,v_2,v_3,v_4$ is linearly independent iff $w_1,w_2,w_3,w_4$ is linarly independent. This concludes the proof.
            \end{solution}
        \end{enumerate}
    \end{problem}

    \begin{problem}
        Suppose that $\{v_1,v_2,\dots,v_n\}$ is linearly independent in $V$. Show that $\{v_1,v_1+v_2,\dots,v_1+v_2+\cdots+v_n\}$ is linearly independent as well.
    \end{problem}
    \begin{solution}
        We know that linear independence of $v_1,\dots,v_n$ says that the only solution to $\sum_i a_iv_i=0$ is $a_i=0$ for all $i$. Now let us suppose that $a_i\in\bR$ satisfies
        \[a_1v_1+a_2(v_1+v_2)+\dots+a_n(v_1+\dots+v_n)=0.\]
        Let us rewrite this as
        \[(a_1+\dots+a_n)v_1+(a_2+\dots+a_n)v_2+\dots+a_nv_n=0.\]
        By linear independence of $v_1,\dots,v_n$, we see that
        \begin{align*}
            a_1+a_2+\dots+a_n &=0,\\
            a_2+\dots+a_n &=0,\\
            &\vdots\\
            a_n &=0.
        \end{align*}
        Using back-substitution we find that the only solution to this system is $a_i=0$ for all $i$. Thus, $v_1,v_1+v_2,\dots,v_1+v_2+\dots+v_n$ are linearly independent.
    \end{solution}

    \begin{problem}
        \leavevmode\begin{enumerate}[(a)]
            \item Show that $V$ is infinite dimensional if and only if it satisfies the following property: for every integer $k>0$, one can find $k$ linearly independent vectors $v_1,\dots,v_k\in V$.
            \begin{solution}
                Recall that the definition of $V$ being infinite dimensional is that there does not exist a finite basis for $V$.

                For one direction, suppose that $V$ is infinite dimensional. Thus we know that there does not exist a finite basis for $V$. We prove that for every $k>0$ we can find $k$ linearly independent vectors $v_1,\dots,v_k\in V$ by induction. For the base case $k=1$, we can certainly find a nonzero vector in $V$ since $V$ cannot be the zero vector space. Now suppose we have found $k-1$ linearly independent vectors $v_1,\dots,v_{k-1}\in V$. Because $v_1,\dots,v_{k-1}$ is not a basis (because $V$ is not supposed to have a finite basis), but is linearly independent, it must be that they do not span $V$. Hence we can let $v_k$ be any element of $V-\Span(v_1,\dots,v_{k-1})$ to complete the induction.

                Now suppose that $V$ is finite dimensional and has dimension $n$. Then any set of $n+1$ vectors must be linearly dependent. This completes the proof.
            \end{solution}
            \item Show that the vector space $\bR^\infty\coloneqq\{\text{ all sequences }(a_1,a_2,a_3,\dots)\text{ of real numbers}\}$ is infinite dimensional.
            \begin{solution}
                The sequences $(1,0,0,0,\dots,), (0,1,0,0,\dots), (0,0,1,0,\dots), \ldots$ gives a sequence of linearly independent sequences in $V$. For any integer $k>0$ we can take the first $k$ terms of this sequence to get $k$ linearly independent vectors in $\bR^\infty$. By part (a) it follows that $\bR^\infty$ is infinite dimensional.
            \end{solution}
            \item Give an example of a subspace of $\bR^\infty$ which is strictly contained in $\bR^\infty$ but is still infinite dimensional.
            \begin{solution}
                Some examples:
                \begin{itemize}
                    \item All sequences whose first term is 0.
                    \item All sequences whose second term is 0.
                    \item All sequences whose first and 42th term are zero.
                    \item All sequences whose first through 42nd term are zero.
                    \item All sequences with finitely many nonzero terms (fun fact: $(1,0,0,0,\dots,), (0,1,0,0,\dots), (0,0,1,0,\dots), \ldots$ is a basis of this space but \textbf{not} a basis of $\bR^\infty$!)
                \end{itemize}
                Can you come up with a cool example that is not like these?

                A non-example is the set $S$ of sequences whose terms are integers. This set fails to be closed under scalar multiplication: $0.5\cdot (1,0,\dots,0)=(0.5,0,\dots,0)$ is not in $S$.
            \end{solution}
        \end{enumerate}
    \end{problem}

    \begin{problem}
        For each positive integer $n$, let \[B_n=\{(-1,1,\dots,1),(1,-1,1,\dots,1),\dots,(1,1,\dots,-1)\}\subseteq F^n.\] That is, $B_n$ is the set of vectors in $F^n$ with one component equal to $-1$ and $n-1$ components equal to 1. 
        \begin{enumerate}[(a)]
            \item Let $F=\bR$. For which $n$ is $B_n$ a basis of $\bR^n$?\footnote{If you need a hint for where to start, try to check whether the vectors in $B_n$ are linearly independent.}
            \begin{solution}
                I claim that the set of $n$ for which $B_n$ is a basis of $\bR^n$ is the set of positive integers except 2.
                
                For $n=1$, $B_n$ consists of $\{-1\}$ which is clearly linearly independent. For $n=2$, $B_n=\{(-1,1),(1,-1)\}$. These vectors are multiples of each other, so cannot be a basis of $\bR^2$ as they are not linearly independent.

                For $n\geq 3$, we will prove that $B_n$ is linearly independent, which proves that $B_n$ is a basis of $\bR^n$ as $|B_n|=n$. To prove that $B_n$ is linearly independent, let us suppose that $a_1,\dots,a_n\in\bR$ such that
                \[a_1(-1,1,\dots,1)+a_2(1,-1,1,\dots,1)+\dots+a_n(1,1,\dots,-1)=0.\]
                This is equivalent to the system of equations
                \begin{align*}
                    -a_1+a_2+\dots+a_n &= 0\\
                    a_1-a_2+\dots+a_n &= 0\\
                    &\vdots\\
                    a_1+a_2\dots-a_n &= 0.
                \end{align*}
                To solve this system, let's note that if we add these equations together we get
                \[(n-2)(a_1+\dots+a_n) = 0,\]
                which implies that $a_1+\dots+a_n=0$, since $n-2\neq 0$ for $n\geq 3$. Now subtracting this last equation by each of the equations in the original system, we see that $2a_i=0$ for all $i$. Hence $a_i=0$ for all $i$ is the only solution to the system, therefore $B_n$ is linearly independent.
            \end{solution}
            \item Let $F=\bF_3$. Show that if $n$ is of the form $3k+2$ for some $k\in\bZ_{\geq 0}$ (i.e.\ $n\in\{2,5,8,\dots\}$), then $B_n$ is not a basis of $\bF_3^n$.\footnote{Hint: Try to show that $B_n$ is contained in a proper subspace.}
            \begin{solution}
                (Side note: If $n\not\equiv 2\pmod 3$, then $n-2\neq 0$ in $\bF_3$, so everything in the solution to part (a) continues to hold. So in this case, $B_n$ is a basis of $\bF_3^n$. But this argument is not needed because the problem did not ask to show that $B_n$ is a basis when $n\notin\{2,5,8,\dots\}$.)
                
                Let $n\equiv 2\pmod 3$. The argument from (a) fails because $n-2=0$, so the equation 
                \[(n-2)(a_1+\dots+a_n)=0\]
                says nothing. However, this alone is not enough to conclude that $B_n$ is not a basis, because maybe some other argument might show that $B_n$ is a basis.

                To conclude that $B_n$ is not a basis, it suffices to make the following observation: when $n\equiv 2\pmod 3$, every vector in $B_n$ has the property that the sum of its components is 0. Indeed, each vector has one component equal to $-1$ and $n-1$ components equal to 1, which makes the sum of the components equal to $-1+(n-1)1=n-2=0\in\bF_3$ when $n\equiv 2\pmod 3$. Therefore,
                \[B_n\subseteq U\coloneqq\{(a_1,\dots,a_n)\in\bF_3^n:a_1+\dots+a_n=0\}.\]
                Note that $U$ is a vector space, in fact a proper subspace of $\bF_3^n$. It follows that $\Span B_n\subseteq U$ as well. Hence, $B_n$ does not span $\bF_3^n$, so $B_n$ is not a basis of $\bF_3^n$.

                (By the basis theorem, it follows that this set of vectors must be linearly dependent when $n\equiv 2\pmod 3$. Can anyone find an explicit linear relation?)
            \end{solution}
        \end{enumerate}
    \end{problem}
\end{document}
