\documentclass[11pt,oneside]{amsart}
\usepackage{geometry}
\usepackage{amssymb,parskip,mathtools,microtype}
\usepackage[shortlabels]{enumitem}
\usepackage[most]{tcolorbox}

\definecolor{sol}{rgb}{0.1, 0.3, 0.6}

\newtcolorbox{solution}{enhanced, breakable, colframe=sol, title=Solution}

\theoremstyle{definition}
\newtheorem{problem}{Problem}

\newcommand{\bC}{\mathbb{C}}
\newcommand{\bF}{\mathbb{F}}
\newcommand{\bQ}{\mathbb{Q}}
\newcommand{\bR}{\mathbb{R}}
\newcommand{\bZ}{\mathbb{Z}}
\newcommand{\eps}{\varepsilon}
\newcommand*\colvec[1]{\begin{psmallmatrix}#1\end{psmallmatrix}}
\newcommand*\dcolvec[1]{\begin{pmatrix}#1\end{pmatrix}}

\DeclareMathOperator{\Span}{span}
\let\Re\relax
\DeclareMathOperator{\Re}{Re}
\let\Im\relax
\DeclareMathOperator{\Im}{Im}
\DeclareMathOperator{\im}{im}
\DeclareMathOperator{\rank}{rank}

\title{MATH2211 Spring 2022\\
Problem Set 8}
\author{Due Friday, April 8, 2022 at 11:59 pm}

\begin{document}
    \maketitle
    
    \begin{problem}
        For each of the following matrices $A$, find the eigenvalues of $A$ and a basis for each eigenspace, and say whether $A$ is diagonalizable.
        \begin{enumerate}[(a)]
            \item $A=\begin{pmatrix}
                1&1\\0&1
            \end{pmatrix}$.
            \begin{solution}
                The only eigenvalue of $A$ is 1 and its eigenspace has basis $\colvec{1\\0}$. Since $A$ does not have a full basis of eigenvectors, $A$ is not diagonalizable.
            \end{solution}
            \item $A=\begin{pmatrix}
                1&1\\1&0
            \end{pmatrix}$.
            \begin{solution}
                The eigenvalues of $A$ are $\frac 12(1\pm\sqrt5)$ (the golden ratio and the conjugate golden ratio!). The eigenspace corresponding to $\frac 12(1+\sqrt 5)$ is the span of $\colvec{\frac 12(1+\sqrt 5)\\1}$. The eigenspace corresponding to $\frac 12(1-\sqrt 5)$ is the span of $\colvec{\frac12(1-\sqrt 5)\\2}$.

                Because $A$ has a full basis of eigenvectors, $A$ is diagonalizable.
            \end{solution}
            \item $A=\begin{pmatrix}
                7&4&4\\0&-1&0\\-8&-4&-5
            \end{pmatrix}$.
            \begin{solution}
                The eigenvalues are $-1$ (with multiplicity 2) and 3. The eigenspace of $-1$ is the span of $\colvec{1\\0\\-2}$ and $\colvec{0\\1\\-1}$ and the eigenspace of 3 is the span of $\colvec{1\\0\\-1}$. We have a full basis of eigenvectors, so $A$ is diagonalizable.
            \end{solution}
        \end{enumerate}
    \end{problem}

    \begin{problem}
        Compute the characteristic polynomial of the $n\times n$ matrix
        \[A=\begin{pmatrix}
            0&0&\cdots&0&0&-a_0\\
            1&0&\cdots&0&0&-a_1\\
            \vdots&\vdots&\cdots&\vdots&\vdots&\vdots\\
            0&0&\cdots&1&0&-a_{n-2}\\
            0&0&\cdots&0&1&-a_{n-1}
        \end{pmatrix}.\]
    \end{problem}
    \begin{solution}
        The characteristic polynomial of $A$ is
        \[\begin{split}
            \chi_A(t)=\det(tI_n-A)=\det\begin{pmatrix}
                t&0&\cdots&0&0&a_0\\
                -1&t&\cdots&0&0&a_1\\
                \vdots&\vdots&\cdots&\vdots&\vdots&\vdots\\
                0&0&\cdots&-1&t&a_{n-2}\\
                0&0&\cdots&0&-1&t+a_{n-1}
            \end{pmatrix}.
        \end{split}\]
        There are many ways to compute this determinant. If you cofactor expand along the first row or column, you will end up using recursion to find the determinant. Here's an idea that leads to a non-recursive computation: cofactor expand along the last column. For $1\leq i\leq n$, the matrix $\overline{A_{i,n}}$ is a block matrix consisting of two blocks: the first block is a lower triangular $(i-1)\times (i-1)$ matrix with $t$'s on the diagonal, and the second block is an upper triangular $(n-i+1)\times (n-i+1)$ matrix with $-1$'s on the diagonal. The determinant $\overline{A_{i,n}}$ is therefore $t^{i-1}\cdot (-1)^{n-i+1}$. Therefore,
        \[\begin{split}
            \det(tI_n-A) &= \left( \sum_{i=1}^{n-2} (-1)^{n+i-1}a_{i-1} t^{i-1}(-1)^{n-i+1} \right) + (t+a_{n-1})t^{n-1}\\
            &= \left( \sum_{i=1}^{n-2}a_{i-1}t^{i-1} \right) + (t+a_{n-1})t^{n-1}\\
            &= t^n+a_{n-1}t^{n-1}+a_{n-2}t^{n-2}+\cdots+a_1t+a_0.
        \end{split}\]
        Another successful idea is to row reduce $\det(tI_n-A)$ to compute its determinant.
    \end{solution}

    \begin{problem}
        A \emph{stochastic matrix} is a square matrix of real numbers, all of whose entries are between 0 and 1 (inclusive), and with the property that the numbers in each column add to 1.

        Prove that 1 is an eigenvalue of any stochastic matrix.
    \end{problem}
    \begin{solution}
        Let $A$ be an $n\times n$ stochastic matrix. Each column of $A-I$ adds to 0, therefore the image of $A-I$ is contained in the subspace of $F^n$ of vectors whose entries add to 0. Hence $A-I$ is not surjective, therefore not invertible, therefore 1 is an eigenvalue of $A$.

        One may also notice that $(1,1,\dots,1)$ is a left eigenvector of $A$, meaning that 1 is an eigenvalue of $^tA$, and use the fact that $A$ and $^tA$ share ths same eigenvalues. (Although this isn't a result we have proved in the course\dots)

        Here's how the two solutions are connected: Saying that the image of $A-I$ is contained in the subspace of vectors whose entries add to 0 is the same as saying that the image of $A-I$ is contained in the kernel of the row vector $(1,1,\dots,1)$. Hmm\dots
    \end{solution}
    
    \begin{problem}
        Let $T\colon V\to V$ be a linear transformation and $B\colon V\to V$ be an invertible linear transformation. Prove that
        \[\chi_T(t)=\chi_{B^{-1}TB}(t). \]
    \end{problem}
    \begin{solution}
        Notice that
        \[B^{-1}(tI-T)B=B^{-1}tIB-B^{-1}TB=tI-B^{-1}TB.\]
        Therefore,
        \[\chi_T(t)=\det(tI-T)=\det(B^{-1}(tI-T)B)=\det(tI-B^{-1}TB)=\chi_{B^{-1}TB}(t),\]
        where we used the multiplicativity of determinant for the second equality.
    \end{solution}

    \begin{problem}
        Prove that the eigenvalues of an upper triangular square matrix are the numbers on the diagonal.
    \end{problem}
    \begin{solution}
        Let $A=(a_{ij})_{1\leq i,j\leq n}$ be an upper triangular square matrix. Then $tI-A$ is also an upper triangular square matrix, and we know from the previous problem set that its determinant is the product of the entries on its diagonal. But the entires on its diagonal are $t-a_{ii}$. Therefore, $\chi_A(t)=\prod_{i=1}^n (t-a_{ii})$, meaning that $a_{ii}$ are the eigenvalues of $A$.
    \end{solution}

    \begin{problem}
        In this problem you will be working out the derivation of the closed form of the Fibonacci sequence that I showed at the beginning of the semester.
        
        Let $\bR^\infty$ be the vector space of infinite sequences of real numbers. Define the left shift operator $S\colon \bR^\infty\to\bR^\infty$ by
        \[S(a_0,a_1,a_2,\dots)=(a_1,a_2,\dots).\]
        \begin{enumerate}[(a)]
            \item Because $\bR^\infty$ is infinite-dimensional (in fact, uncountably-dimensional), weird things can happen. Prove that \emph{every real number} is an eigenvalue of $S$. Also find the corresponding eigenvector for an arbitrary eigenvalue $\lambda\in\bR$.
            \begin{solution}
                The equation $S(a_0,a_1,\dots)=\lambda(a_0,a_1,\dots)$ translates to $a_1=\lambda a_0$, $a_2=\lambda a_1$, $a_3=\lambda a_2$, and so on. For any $\lambda\in\bR$ we have a solution to this equation: $(1,\lambda,\lambda^2,\lambda^3,\dots)$. So any $\lambda$ is an eigenvalue of $S$.
            \end{solution}
            \item Let $V\subseteq\bR^\infty$ be the subspace consisting of Fibonacci-like sequences; that is, consisting of sequences $(a_0,a_1,a_2,\dots)$ satisfying $a_n=a_{n-1}+a_{n-2}$ for all $n\geq 2$. Prove that if $v\in V$, then $Sv\in V$ as well.
            \begin{solution}
                Let $v=(a_0,a_1,\dots)$. The hypothesis that $v\in V$ says that $a_n=a_{n-1}+a_{n-2}$ for all $n\geq 2$. Now $Sv=(a_1,a_2,\dots)$ and the hypothesis that $Sv\in V$ translates to $a_n=a_{n-1}+a_{n-2}$ for $n\geq 3$. But this automatically follows from the hypothesis that $a_n=a_{n-1}+a_{n-2}$ for all $n\geq 2$. Therefore, $Sv\in V$.
            \end{solution}
            \item Part (b) showed that $S$ restricts to a linear map $S|_V\colon V\to V$. What is the dimension of $V$? What are the eigenvalues and eigenvectors of $S|_V$?
            \begin{solution}
                A Fibonacci-like sequence is uniquely determined by its first two terms, and conversely any assignment of the first two terms of a sequence extends uniquely to a Fibonacci-like sequence. Therefore, $V$ is 2-dimensional.

                The eigenvectors of $S|_V$ are the eigenvectors of $S$ which lie in $V$. In part (a) we found that the eigenvectors of $V$ are multiples of vectors of the form $(1,\lambda,\lambda^2,\dots)$. Now the condition that $(1,\lambda,\lambda^2,\dots)\in V$ is the condition that $\lambda^n=\lambda^{n-1}+\lambda^{n-2}$ for all $n\geq 2$. These infinitely many equations are all satisfied iff $\lambda^2=\lambda+1$. There are exactly two values of $\lambda$ which satisfy this quadratic equation: the golden ratio $\phi=(1+\sqrt 5)/2$ and the conjugate golden ratio $\psi=(1-\sqrt 5)/2$. Therefore, $\phi$ and $\psi$ are the eigenvalues of $S|_V$ and the corresponding eigenvectors are $(1,\phi,\phi^2,\dots)$ and $(1,\psi,\psi^2,\dots)$.
            \end{solution}
            \item Derive the closed form for the Fibonacci sequence $a_0=0,a_1=1; a_n=a_{n-1}+a_{n-2}$ for $n\geq 2$, using part (c).
            \begin{solution}
                We have a basis of eigenvectors, so let's write $(0,1,1,2,3,5,\dots)$ as a linear combination of $(1,\phi,\phi^2,\dots)$ and $(1,\psi,\psi^2,\dots)$, that is, let us find $a$ and $b$ such that
                \[(0,1,1,2,\dots)=a(1,\phi,\phi^2,\dots)+b(1,\psi,\psi^2,\dots).\]
                Because these eigenvectors are a basis of $V$, we know such $a$ and $b$ must exist and are unique. The equation above expands to infinitely many equations in $a$ and $b$, among which are $a-b=0$ and $\phi a+\psi b=1$. This has the solution $a=\frac 1{\sqrt 5}$ and $b=-\frac 1{\sqrt5}$, so we know this pair $(a,b)$ must solve the other equations as well. Therefore,
                \[\begin{split}
                    (F_n,F_{n+1},\dots) &=S^n(0,1,1,2,\dots)\\
                    &= S^n\left(\frac 1{\sqrt5}(1,\phi,\phi^2,\dots)-\frac 1{\sqrt5}(1,\psi,\psi^2,\dots)\right)\\
                    &=\frac 1{\sqrt5}(\phi^n,\phi^{n+1},\dots)-\frac 1{\sqrt5}(\psi^n,\psi^{n+1},\dots).
                \end{split}\]
                So
                \[F_n=\frac 1{\sqrt 5}\phi^n-\frac 1{\sqrt5}\psi^n\text{ for all $n\geq 0$}.\]
            \end{solution}
            \item To convince you that this ``abstract'' proof is not actually that abstract, here is a final exercise. Let $\varphi\colon\bR^2\to V$ be the isomorphism sending $(x,y)$ to the Fibonacci-like sequence whose first two terms are $a_0=x$ and $a_1=y$. Then $\varphi^{-1}\colon V\to\bR$ is the projection $(a_0,a_1,\dots,)\mapsto (a_0,a_1)$. Thus, the composition
            \[\bR^2\xrightarrow{\varphi}V\xrightarrow S V\xrightarrow{\varphi^{-1}}\bR^2\]
            is a linear transformation from $\bR^2\to\bR^2$, i.e.\ a $2\times 2$ matrix. What is this matrix?
            \begin{solution}
                Let's see where $(1,0)$ and $(0,1)$ get mapped to:
                \begin{align*}
                    (1,0) &\xmapsto \varphi (1,0,1,1,2,3,\dots)\xmapsto S (0,1,1,2,3,\dots)\xmapsto{\varphi^{-1}} (0,1) \\
                    (0,1) &\xmapsto\varphi (0,1,1,2,3,\dots) \xmapsto S (1,1,2,3,\dots)\xmapsto{\varphi^{-1}} (1,1).
                \end{align*}
                So the matrix of $\varphi^{-1}S\varphi$ is
                \[\begin{pmatrix}
                    0&1\\
                    1&1
                \end{pmatrix}.\]
            \end{solution}
        \end{enumerate}
    \end{problem}
\end{document}
