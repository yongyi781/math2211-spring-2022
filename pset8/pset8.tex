\documentclass[11pt,oneside]{amsart}
\usepackage{geometry}
\usepackage{amssymb,parskip,mathtools,microtype}
\usepackage[shortlabels]{enumitem}

\theoremstyle{definition}
\newtheorem{problem}{Problem}

\newcommand{\bC}{\mathbb{C}}
\newcommand{\bF}{\mathbb{F}}
\newcommand{\bQ}{\mathbb{Q}}
\newcommand{\bR}{\mathbb{R}}
\newcommand{\bZ}{\mathbb{Z}}
\newcommand{\eps}{\varepsilon}
\newcommand*\colvec[1]{\begin{psmallmatrix}#1\end{psmallmatrix}}
\newcommand*\dcolvec[1]{\begin{pmatrix}#1\end{pmatrix}}

\DeclareMathOperator{\Span}{span}
\let\Re\relax
\DeclareMathOperator{\Re}{Re}
\let\Im\relax
\DeclareMathOperator{\Im}{Im}
\DeclareMathOperator{\im}{im}
\DeclareMathOperator{\rank}{rank}

\title{MATH2211 Spring 2022\\
Problem Set 8}
\author{Due Friday, April 8, 2022 at 11:59 pm}

\begin{document}
    \maketitle
    
    \begin{problem}
        For each of the following matrices $A$, find the eigenvalues of $A$ and a basis for each eigenspace, and say whether $A$ is diagonalizable.
        \begin{enumerate}[(a)]
            \item $A=\begin{pmatrix}
                1&1\\0&1
            \end{pmatrix}$.
            \item $A=\begin{pmatrix}
                1&1\\1&0
            \end{pmatrix}$.
            \item $A=\begin{pmatrix}
                7&4&4\\0&-1&0\\-8&-4&-5
            \end{pmatrix}$.
        \end{enumerate}
    \end{problem}

    \begin{problem}
        Compute the characteristic polynomial of the $n\times n$ matrix
        \[A=\begin{pmatrix}
            0&0&\cdots&0&0&-a_0\\
            1&0&\cdots&0&0&-a_1\\
            \vdots&\vdots&\cdots&\vdots&\vdots&\vdots\\
            0&0&\cdots&1&0&-a_{n-2}\\
            0&0&\cdots&0&1&-a_{n-1}
        \end{pmatrix}.\]
    \end{problem}

    \begin{problem}
        A \emph{stochastic matrix} is a square matrix of real numbers, all of whose entries are between 0 and 1 (inclusive), and with the property that the numbers in each column add to 1.

        Prove that 1 is an eigenvalue of any stochastic matrix.
    \end{problem}
    
    \begin{problem}
        Let $T\colon V\to V$ be a linear transformation and $B\colon V\to V$ be an invertible linear transformation. Prove that
        \[\chi_T(t)=\chi_{B^{-1}TB}(t). \]
    \end{problem}

    \begin{problem}
        Prove that the eigenvalues of an upper triangular square matrix are the numbers on the diagonal.
    \end{problem}

    \begin{problem}
        In this problem you will be working out the derivation of the closed form of the Fibonacci sequence that I showed at the beginning of the semester.
        
        Let $\bR^\infty$ be the vector space of infinite sequences of real numbers. Define the left shift operator $S\colon \bR^\infty\to\bR^\infty$ by
        \[S(a_0,a_1,a_2,\dots)=(a_1,a_2,\dots).\]
        \begin{enumerate}[(a)]
            \item Because $\bR^\infty$ is infinite-dimensional (in fact, uncountably-dimensional), weird things can happen. Prove that \emph{every real number} is an eigenvalue of $S$. Also find the corresponding eigenvector for an arbitrary eigenvalue $\lambda\in\bR$.
            \item Let $V\subseteq\bR^\infty$ be the subspace consisting of Fibonacci-like sequences; that is, consisting of sequences $(a_0,a_1,a_2,\dots)$ satisfying $a_n=a_{n-1}+a_{n-2}$ for all $n\geq 2$. Prove that if $v\in V$, then $Sv\in V$ as well.
            \item Part (b) showed that $S$ restricts to a linear map $S|_V\colon V\to V$. What is the dimension of $V$? What are the eigenvalues and eigenvectors of $S|_V$?
            \item Derive the closed form for the Fibonacci sequence $a_0=0,a_1=1; a_n=a_{n-1}+a_{n-2}$ for $n\geq 2$, using part (c).
            \item To convince you that this ``abstract'' proof is not actually that abstract, here is a final exercise. Let $\varphi\colon\bR^2\to V$ be the isomorphism sending $(x,y)$ to the Fibonacci-like sequence whose first two terms are $a_0=x$ and $a_1=y$. Then $\varphi^{-1}\colon V\to\bR$ is the projection $(a_0,a_1,\dots,)\mapsto (a_0,a_1)$. Thus, the composition
            \[\bR^2\xrightarrow{\varphi}V\xrightarrow S V\xrightarrow{\varphi^{-1}}\bR^2\]
            is a linear transformation from $\bR^2\to\bR^2$, i.e.\ a $2\times 2$ matrix. What is this matrix?
        \end{enumerate}
    \end{problem}
\end{document}
