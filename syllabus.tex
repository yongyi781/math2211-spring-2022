\documentclass[11pt,oneside]{amsart}
\usepackage{geometry}
\usepackage{parskip,hyperref}

\pagestyle{empty}

\title{Course information for MATH2211 (Spring 2022)\\
    Honors Linear Algebra}

\begin{document}
\maketitle

\textbf{Instructor:} Yongyi Chen\\
\textbf{Email:} \href{mailto:yongyi.chen@bc.edu}{\texttt{yongyi.chen@bc.edu}}

\textbf{Lectures:} MWF 12:00 pm--12:50 pm in Campion Hall 9\\
\textbf{Homework:} Weekly, due on Wednesdays at 11:59 pm.

\textbf{Office:} Maloney 532\\
\textbf{Office hours:} (tentative) Mondays 2-3 pm in Maloney 532, Tuesdays 4-6 pm over Zoom

\section{Course information}
\subsection*{Course website}
On Canvas. There you will find homework assignments, homework solutions, and supplemental course materials.

\subsection*{Course format}
In person. 1 office hour is provided in person in my office and 2 office hours will be over Zoom.

\subsection*{Textbooks}
There is no official textbook for the course. We will follow the notes written by Prof. Keane for the Spring 2019 version of the course. The notes can be found on Canvas. For a few topics we will follow \emph{Linear Algebra Done Right} by Sheldon Axler, available on SpringerLink.

In addition to these course notes, you may find the following textbooks useful:
\begin{itemize}
    \item \emph{Linear Algebra Done Wrong} by Sergei Treil. Available free online.
    \item An experimental linear algebra zyBook (more information on Canvas).
\end{itemize}

\subsection*{Homework}
There will be weekly homework, due on Wednesdays at 11:59 pm. Because homework solutions will be posted on Canvas, late homework will not be accepted. To submit your homework, upload a single PDF file to Gradescope (accessible from within the Canvas assignment page as well).

You are encouraged to collaborate on homework with your classmates, but the work that you turn in must be your own and must be written in your own words.  Working together is good; copying somebody else’s work is plagiarism.

One of the primary differences between this course and its non-honors variant is the emphasis on careful mathematical reasoning and proof.  As such, writing style counts as much as having the right answer (often you will be told the answer and asked to justify it).  Homework solutions must be written in complete sentences, and must be clear, concise, and readable.  A correct but poorly expressed solution will not receive full credit.

Typesetting your homework using LaTeX is strongly encouraged, but not required.

\subsection*{Exams and grading}
There will be two in-class exams (50 minutes each) and a final (120 minutes). Final grades will be determined by a weighted average of homework and exam scores.  Homework counts for 20\%, each in-class exam counts for 20\%, and the final counts for 40\%.

All exams will be given in class.

\subsection*{Academic integrity}
Cheating of any kind will result in a failing grade for the course and referral to the Dean’s office for disciplinary action.  For more information on academic integrity see \url{https://www.bc.edu/integrity}.

\section{List of topics}
\begin{enumerate}
    \item Warm-up
    \begin{itemize}
        \item Sets, fields, functions, induction, complex numbers
    \end{itemize}

    \item Vector spaces
    \begin{itemize}
        \item Subspaces, span, and linear independence
        \item Bases and dimension
    \end{itemize}

    \item Linear transformations
    \begin{itemize}
        \item Linear transformations and matrices
        \item Kernels, images, and invertibility
        \item Products, quotients, and duals
    \end{itemize}

    \item Gaussian elimination
    \begin{itemize}
        \item Systems of linear equations
        \item Row reduction and elementary matrices
        \item Computing inverses
    \end{itemize}

    \item Determinants
    \begin{itemize}
        \item Determinants and invertibility
        \item Expansion by minors
        \item Cramer's rule
    \end{itemize}

    \item Spectral theory
    \begin{itemize}
        \item Polynomials
        \item Eigenvectors and eigenvalues
        \item The characteristic polynomial
        \item Diagonalizing matrices
    \end{itemize}

    \item Other topics, as time permits:
    \begin{itemize}
        \item Inner product spaces
        \item The Cayley-Hamilton theorem and Jordan normal form
    \end{itemize}
\end{enumerate}

\end{document}